\section{Okoř}
\aindex{Lidovky!Okoř}
%Řádkování 1.5
\onehalfspacing

\ch{D}Na Okoř je cesta jako žádná ze sta \ch{A7} vroubená je stroma \ch{D} ma \\
\ch{D} když du po ní v létě samoten ve světě \ch{A7}  sotva pletu noha \ch{D} ma \\
\ch{G} na konci té cesty \ch{D} trnité \ch{E}  stojí krčma jako \ch{A7}  hrad \\
\ch{D} tam zapadlí trampi hladoví a sešlí \ch{A7}  začli sobě noto \ch{D} vat \\

\ch{D} Na hradě Okoři \ch{A7}  světla už nehoří \ch{D} bílá paní \ch{A7}  šla už dávno \ch{D} spát \\
\ch{D} ona měla ve zvyku \ch{A7}  podle svého budíku \ch{D} o půlnoci \ch{A7}  chodit straší \ch{D} vat \\
\ch{G}  od těch dob co jsou tam \ch{D} trampové \ch{E}  nesmí z hradu \ch{A7}  pryč \\
\ch{D} a tak dole v podhradí \ch{A7}  se šerifem dovádí \ch{D} on ji sebral \ch{A7}  od komnaty \ch{D} klíč \\

%Začátek dvousloupcového režimu
%\begin{multicols}{2}
%Řádkování 1
\singlespacing

\sloka{}
Jednoho dne z rána roznesla se zpráva že byl Okoř vykraden \\
nikdo neví dodnes kdo to tenkrát odnes nikdo nebyl dopaden \\
šerif hrál celou noc mariáš s bílou paní v kostnici \\
místo aby hlídal zuřivě ji líbal dostal z toho zimnici \\

%\end{multicols}