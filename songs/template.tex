\section{Když náš táta hrál}
\aindex{Greenhorns!Když náš táta hrál}
%Řádkování 1.5
\onehalfspacing

Když \ch{G}jsem byl chlapec malej, tak metr nad zemí,\\
\ch{C}scházeli se farmáři tam \ch{G}u nás v přízemí,\\
\ch{G}mezi nima můj táta u piva sedával\\
\ch{D7}a tu svoji nejmilejší \ch{G}hrál.\\
\textbf{C, G, D7, G}\\

\refren
Kví, kví, kví, kví k výčepu najdu cestu třeba poslepu, \\
kví, kví, kví, kví, kví, kví, \\
kví, k výčepu najdu cestu třeba poslepu\\

Teď už jsem chlap jak hora, šest stop a palců pět,\\
už jsem prošel celý Státy a teď táhnu zpět,\\
kdybych si ale ve světě moh' ještě něco přát,\\
tak zase slyšet svýho tátu hrát.\\

\refren
Kví, kví, kví, kví k výčepu najdu cestu třeba poslepu, \\
kví, kví, kví, kví, kví, kví, \\
kví, k výčepu najdu cestu třeba poslepu\\

Ta písnička mě vedla mým celým životem,\\
když jsem se toulal po kolejích, žebral za plotem,\\
a když mi bylo nejhůř, tak přece jsem se smál,\\
když jsem si vzpomněl, jak náš táta hrál.\\

\refren
Kví, kví, kví, kví k výčepu najdu cestu třeba poslepu, \\
kví, kví, kví, kví, kví, kví, \\
kví, k výčepu najdu cestu třeba poslepu\\

To všechno už je dávno, táta je pod zemí,\\
když je noc a měsíc, potom zdá se mi,\\
jako bych od hřbitova, kam tátu dali spát,\\
zase jeho píseň slyšel hrát.\\

\refren
Kví, kví, kví, kví k výčepu najdu cestu třeba poslepu, \\
kví, kví, kví, kví, kví, kví, \\
kví, k výčepu najdu cestu třeba poslepu\\

\textbf{Alternativní refrén 1.}\\
A proto táto, táto, hraj na banjo dál\\
ty jsi bejval všech farmářů král\\

\textbf{Alternativní refrén 2.}\\
Ty jsi ta nejkrásnější holka z JZD\\
to říkají všichni sousedé\\
