\section{Hlídač krav}
\aindex{Jaromír Nohavica!Hlídač krav}
%Řádkování 1.5
\onehalfspacing

\ch{D}Když jsem byl malý, říkali mi naši: \\
"Dobře se uč a jez chytrou kaši, \\
\ch{G}až jednou vyrosteš, \ch{A7}budeš doktorem \ch{D}práv. \\
Takový doktor sedí pěkně v suchu, \\
bere velký peníze a škrábe se v uchu," \\
\ch{G}já jim ale na to řek': "Ch\ch{A7}ci být hlídačem \ch{D}krav."\\

\refren
Já chci \ch{D}mít čapku s bambulí nahoře, \\
jíst kaštany a mýt se v lavoře, \\
\ch{G}od rána po celý \ch{A7}den zpívat si \ch{D}jen,\\
zpívat si: pam pam pam \textbf{D, G, A7 ...}\\


%Začátek dvousloupcového režimu
%\begin{multicols}{2}
%Řádkování 1
\singlespacing

\sloka{}
K vánocům mi kupovali hromady knih,\\
co jsem ale vědět chtěl, to nevyčet' jsem z nich:\\
nikde jsem se nedozvěděl, jak se hlídají krávy.\\
Ptal jsem se starších a ptal jsem se všech,\\
každý na mě hleděl jako na pytel blech,\\
každý se mě opatrně tázal na moje zdraví.\\

\refren
Já chci mít čapku s bambulí nahoře, \\
jíst kaštany a mýt se v lavoře, \\
od rána po celý den zpívat si jen,\\
zpívat si: pam pam pam \\


\sloka{}
Dnes už jsem starší a vím, co vím,\\
mnohé věci nemůžu a mnohé smím,\\
a když je mi velmi smutno, lehnu si do mokré trávy.\\
S nohama křížem a s rukama za hlavou\\
koukám nahoru na oblohu modravou,\\
kde se mezi mraky honí moje strakaté krávy.\\

\refren
 ...... pa da da dam [D]
%\end{multicols}