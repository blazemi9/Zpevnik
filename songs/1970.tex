\section{1970}
\aindex{Chinaski!1970}
%Řádkování 1.5
\onehalfspacing

\ch{C}Nevím jestli je to z\ch{Dmi}nát 
možná by bylo lepší l\ch{C}hát \\
jsem silnej ročník \ch{F}sedmdesát 
tak začni počí\ch{C}tat \\
nechci tu hloupě vzpomí\ch{Dmi}nat 
koho to taky zají\ch{C}má \\
silnej ročník \ch{F}sedmdesát 
tak začni počí\ch{C}tat \\
tenkrát tu bejval jinej s\ch{Dmi}tát 
a já byl blbej na kva\ch{C}drát \\
jsem silnej ročník \ch{F}sedmdesát 
tak třeba napří\ch{G}klad \\

\refren
\ch{G}Naši mi vždycky říka\ch{Ami}li \\
jen nehas co tě nepá\ch{F}lí \\
jakej pán takovej \ch{C}krám \\
\ch{G}Naši mi vždycky říka\ch{Ami}li \\
co můžeš sleduj z povzdá\ch{F}lí \\
a nikdy nebojuj \ch{C}sám\\

%Začátek dvousloupcového režimu
%\begin{multicols}{2}
%Řádkování 1
\singlespacing

\sloka{}
Nevím jestli je to znát 
možná by bylo lepší lhát \\
jsem silnej ročník sedmdesát 
nemoch jsem si vybírat \\
tak mi to přestaň vyčítat 
naříkat co jsem za případ \\
jsem silnej ročník sedmdesát 
a možná že jsem rád \\

\refren
Naši mi vždycky říkali 
jen nehas co tě nepálí \\
jakej pán takovej krám 
Naši mi vždycky říkali \\
co můžeš sleduj z povzdálí 
a nikdy nebojuj sám\\

\sloka{}
Čas pádí, čas letí 
těžko ta léta vrátíš zpět \\
a tak i Husákovy děti 
dospěly do Kristových let \\
Čas pádí, čas letí 
a je to zvláštní svět \\
a tak i Husákovy děti 
dospěly do Kristových let \\

%\end{multicols}