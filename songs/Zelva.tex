\section{Želva}
\aindex{Olympic!Želva}
%Řádkování 1.5
\onehalfspacing

\ch{G}Ne moc \ch{C}snadno se \ch{G}želva \ch{C}po dně \ch{G}honí,\\
\textbf{F, C, F, C}\\
velmi \ch{C}radno je \ch{G}plavat \ch{C}na dno za \ch{G}ní,\\
\textbf{F, C, F}\\
jenom \ch{D}počkej, až se zeptá, na to, \ch{Emi}co tě v mozku lechtá,\\
\ch{G}nic se \ch{C}neboj a \ch{G}vem si \ch{C}něco od \ch{G}ní.\\
\textbf{F, C, F}\\

\sloka
Abych zabil dvě mouchy jednou ranou,\\
želví nervy od želvy schovám stranou,\\
jednu káď dám zvlášť pro tebe, a pak aspoň dvě pro sebe.\\
Víš má drahá a zbytek je pod vanou.\\

\refren
\ch{G}Když si \ch{D}někdo \ch{C}pozor nedá\ch{G},\\ 
jak se \ch{D}vlastně \ch{C}želva hledá\ch{G},\\
\ch{C}ona,ho na \ch{A7}něco nachytá\ch{D7}, \\
\ch{C}i když si to \ch{A7}pozděj’c vyčítá\ch{D7}.\\

\sloka
Ne moc lehce se želva po dně honí,\\
ten, kdo nechce, pak často(brzo) slzy roní,\\
jeho úsměv se vytratí a to se mu nevyplatí,\\
má se nebát želev a spousty vodní. \\

\refren
Když si někdo pozor nedá, \\
jak se vlastně želva hledá,\\
ona,ho na něco nachytá, \\
i když si to pozděj’c vyčítá.\\

%Začátek dvousloupcového režimu
%\begin{multicols}{2}
%Řádkování 1
%\singlespacing


%\end{multicols}